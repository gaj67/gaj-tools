\documentclass[a4paper]{article}
\usepackage[a4paper,margin=25mm]{geometry}
\usepackage{graphicx,subcaption}
\usepackage{amsmath,amsfonts}
\usepackage{qtree}
\usepackage{tikz}
\usepackage{varwidth}

\title{Latent-class Hidden Markov Model}
\author{G.A. Jarrad}
\usepackage{accents}
\newcommand{\rvec}[1]{\accentset{\leftarrow}{#1}}

\begin{document}
\maketitle
\numberwithin{equation}{section}
\numberwithin{figure}{section}
\numberwithin{table}{section}
\section{Model}
We assume that a sequence $\rvec{x}=(x_1,x_2,\ldots,x_T)\in{\cal X}^T$ of
observations is drawn from some latent class $k\in{\cal K}=\{1,2,\ldots,K\}$
 according to a hidden Markov model (HMM) with corresponding state sequence
$\rvec{s}=(s_1,s_2,\ldots,s_T)\in\mathscr{S}_k$ with states 
${\cal S}_k=\{1,2,\ldots,S_k\}$.
The joint model then factors as
\begin{eqnarray}
p(k,\rvec{s},\rvec{x}\,|\,\Theta) & = &
p(k\,|\,\Theta)\,p(\rvec{s}\,|\,k,\Theta)\,p(\rvec{x}\,|\,\rvec{s},k,\Theta)\,.
\end{eqnarray}
Now, since $k$ and $\rvec{s}$ are unknown, we might consider the marginal
likelihood
\begin{eqnarray}
\ell(\Theta) & = & p(\rvec{x}\,|\,\Theta)
~=~\sum_{k=1}^K\sum_{\rvec{s}\in\mathscr{S}_k}
p(k,\rvec{s},\rvec{x}\,|\,\Theta)\,.
\end{eqnarray}
For a dataset $\mathbb{X}=(\rvec{x}_1,\rvec{x}_2,\ldots,\rvec{x}_D)$ of
independent observations, we then obtain the data log-likelihood
\begin{eqnarray}
L(\Theta) & = & \ln p(\mathbb{X}\,|\,\Theta)
~=~\sum_{d=1}^D \ln\ell_d(\Theta)
\nonumber\\
& = & \sum_{d=1}^D\ln\sum_{k=1}^K\sum_{\rvec{s}\in\mathscr{S}_k}
p(k,\rvec{s},\rvec{x}_d\,|\,\Theta)\,;
\end{eqnarray}
this could, for example, be maximised to obtain the optimal parameters
$\Theta^*$.

Alternatively, we could compute the per-datum expected log-likelihood
\begin{eqnarray}
q(\Theta) & = & E_{k,\rvec{s}\,|\,\rvec{x},\Theta}\left[\ln
p(k,\rvec{s},\rvec{x}\,|\,\Theta)\right]
\nonumber\\
& = & \sum{k=1}^K p(k\,|\,\rvec{x},\Theta)
\sum_{\rvec{s}\in\mathscr{S}_k} p(\rvec{s}\,|\,k,\rvec{x},\Theta)
\ln p(k,\rvec{s},\rvec{x}\,|\,\Theta)
\,,
\end{eqnarray}
where
\begin{eqnarray}
p(k\,|\,\rvec{x},\Theta) & = & 
\frac{p(k\,|\,\Theta)\,p(\rvec{x}\,|\,k,\Theta)}
{\sum_{k'=1}^K p(k'\,|\,\Theta)\,p(\rvec{x}\,|\,k',\Theta)}\,,
\\
p(\rvec{x}\,|\,k,\Theta) & = & \sum_{\rvec{s}\in\mathscr{S}}
p(\rvec{s}\,|\,k,\Theta)\,p(\rvec{x}\,|\,\rvec{s},k,\Theta)
\end{eqnarray}
and
\begin{eqnarray}
p(\rvec{s}\,|\,k,\rvec{x},\Theta) & = & 
\frac{p(\rvec{s}\,|\,k,\Theta)\,p(\rvec{x}\,|\,\rvec{s},k,\Theta)}
{p(\rvec{x}\,|\,k,\Theta)}\,.
\end{eqnarray}
This gives rise to the expected complete--data log-likelihood
\begin{eqnarray}
Q(\Theta) & = & \sum_{d=1}^D \ln q_d(\Theta)
\nonumber\\
& = & \sum_{d=1}^D\sum{k=1}^K p(k\,|\,\rvec{x}_d,\Theta)
\sum_{\rvec{s}\in\mathscr{S}_k} p(\rvec{s}\,|\,k,\rvec{x}_d,\Theta)
\ln p(k,\rvec{s},\rvec{x}_d\,|\,\Theta)
\,.
\end{eqnarray}

As a second alternative, we might instead utilise the expectation--maximisation
(EM) framework and compute
\begin{eqnarray}
Q(\Theta,\Theta') & = & \sum_{d=1}^D
E_{k,\rvec{s}\,|\,\rvec{x}_d,\Theta'}\left[\ln
p(k,\rvec{s},\rvec{x}_d\,|\,\Theta)\right] \nonumber\\
& = & \sum_{d=1}^D\sum{k=1}^K p(k\,|\,\rvec{x}_d,\Theta')
\sum_{\rvec{s}\in\mathscr{S}_k} p(\rvec{s}\,|\,k,\rvec{x}_d,\Theta')
\ln p(k,\rvec{s},\rvec{x}_d\,|\,\Theta)
\,,
\end{eqnarray}
which gives rise to the iterative sequence of parameter updates
$\Theta^{(1)}\rightarrow\Theta^{(2)}\rightarrow\cdots\rightarrow\Theta^*$,
where
\begin{eqnarray}
\Theta^{(k+1)} = \arg\max_{\Theta} Q(\Theta,\Theta^{(k)})\,.
\end{eqnarray}
%%%%%%%%%%%%%%%%%%%%%%%%%%%%%%%%%%%%
\end{document}
