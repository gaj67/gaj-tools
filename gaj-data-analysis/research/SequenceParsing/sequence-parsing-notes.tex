\documentclass{report}
\usepackage{qtree}
\begin{document}
Consider an ordered sequence of tokens, 
${\bf\tau}=(\tau_1,\tau_2,...,\tau_n)$.
A parse of this sequence provides, under one interpretation,
a nested partitioning of the tokens, e.g. either
\\
\begin{figure}[h]
\centering
\{
 \{{\tt The\, cat}\}\,
 \{{\tt sat}\,
  \{{\tt on}\,
   \{{\tt the\, mat}\}
  \}
 \}
\}\,,
\caption{Representing the parse as a bracketing of the tokens.}
\label{fig:bracketing}
\end{figure}
\\
or
\\
\begin{figure}[h]
\centering
\underline{
   \underline{{\tt The\, cat}}\,\,\,\strut
   \underline{{\tt sat}\,
         \underline{{\tt on}\,\,\strut
            \underline{{\tt the\, mat}}
         }\strut
   }
}\,.
\caption{Representing the parse as a collection of token sub-sequences.}
\label{fig:nesting}
\end{figure}
\\
Under another interpretation, the parse forms a tree of nodes, e.g.
\begin{figure}[h]
\centering
\Tree [  [ The cat ]  [ sat   [ on  [ the mat ] ] ] ] \,.
\caption{Representing the parse as a tree of nodes.}
\label{fig:parse-tree}
\end{figure}
\\
Thus, a parse tree can be characterised by a set of nodes and 
a set of relations or rules linking these nodes. The parse nodes
are divided into leaf nodes and derived nodes. 
A {\em leaf} node represents all of the known information about a 
token, including the position of that token in the token sequence.
A {\em derived} node represents a combination of a contiguous 
sub-sequence of leaf and/or derived nodes.

In order to define node contiguity, first observe from 
Figure~\ref{fig:parse-tree} that each node $\nu$ spans some contiguous
sub-sequence of tokens, e.g. $(\tau_i,\tau_{i+1},...,\tau_{i+k})$.
Let $beta(\nu)$ be the index of the first token in the sub-sequence,
e.g. $\beta(\nu)=i$, and let $\varepsilon(\nu)$ be the index of the 
last token, e.g. $\varepsilon(\nu)=i+k$. Then the {\em span}
of node $\nu$ is defined as the set of token indices
\[
\sigma(\nu)=\{j\in{\cal I}_n
\;|\;\beta(\nu)\le j\le\varepsilon(\nu)\}\,,
\]
where ${\cal I}_n=\{1,2,...,n\}$.
Hence, for two arbitrary nodes $\nu_1$ and $\nu_2$ to 
be said to be {\em non-overlapping}, 
their spans must be mutually exclusive,
i.e. obey $\sigma(\nu_1)\bigcap\sigma(\nu_2)=\{\}$.
Furthermore, for $\nu_1$ and $\nu_2$ to be {\em adjacent}, their spans
must further obey either $\beta(\nu_1)=\varepsilon(\nu_2)+1$
or $\beta(\nu_2)=\varepsilon(\nu_1)+1$.
Finally, an arbitrary sequence $(\nu_1,\nu_2,...,\nu_m)$ of nodes
is {\em contiguous} if and only if
\[
\beta(\nu_{i+1})=\varepsilon(\nu_i)+1,\, \forall i\in{\cal I}_{m-1}\,.
\]

Continuing now from previous remarks, a parse tree ${\cal P}$ may be 
thought of as a set of node combination rules of the form
\[
\nu_1\;\nu_2\;...\;\nu_m \stackrel{\rho}{\rightarrow} \nu\;.
\]
In order to be a valid rule, 
the sequence ${\pi}(\rho)=(\nu_1,\nu_2,...,\nu_m)$ of 
so-called {\em predecessor} nodes must be contiguous, such that the 
resulting derived node $\delta(\rho)=\nu$ has span
\[
\sigma(\nu)=\bigcup_{i=1}^{m}\sigma(\nu_i)\;,
\]
where $\beta(\nu)=\beta(\nu_1)$ and 
$\varepsilon(\nu)=\varepsilon(\nu_m)$.
In effect, the predecessor nodes partition the 
tokens spanned by $\nu$ into mutually exclusive sub-sequences.

Taken together, the set ${\cal P}$ of rules forms a parse
of the whole token sequence, i.e.
\[
\exists\rho\in{\cal P},\,\sigma(\delta(\rho)) = {\cal I}_n\,.
\]
Furthermore, the rules must
partition the token sequence into a tree of nested sub-sequences, i.e. 
\[
\forall \rho_1\in{\cal P},\,
\forall \rho_2\in{\cal P}\backslash\{\rho_1\}
,\,
\sigma(\delta(\rho_1))\bigcap\sigma(\delta\rho_2))\ne\{\}
\Rightarrow\sigma(\delta(\rho_1))\subset
\sigma(\delta(\rho_2))\mbox{ or }
\sigma(\delta(\rho_2))\subset\sigma(\delta(\rho_1))\,.
\]

\end{document}
